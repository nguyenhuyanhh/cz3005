\documentclass[11pt]{report}
    
% report format
\usepackage[utf8]{inputenc}
\usepackage[a4paper, margin=2.54cm]{geometry}
\usepackage{setspace}
\onehalfspacing{}

\usepackage{graphicx}
\graphicspath{{../images/}}
\usepackage{listings}
\usepackage{color}

% some configurations for the listings package
\definecolor{codegreen}{rgb}{0,0.6,0}
\definecolor{codegray}{rgb}{0.5,0.5,0.5}
\definecolor{codepurple}{rgb}{0.58,0,0.82}
\definecolor{backcolour}{rgb}{0.95,0.95,0.92}
\lstdefinestyle{mystyle}{
    backgroundcolor=\color{backcolour},
    commentstyle=\color{codegreen},
    keywordstyle=\color{magenta},
    numberstyle=\tiny\color{codegray},
    stringstyle=\color{codepurple},
    basicstyle=\footnotesize\ttfamily,
    breakatwhitespace=false,
    breaklines=true,
    captionpos=b,
    keepspaces=true,
    % numbers=left,
    numbersep=5pt,
    showspaces=false,
    showstringspaces=false,
    showtabs=false,
}
\lstset{style=mystyle}

% actual report
\begin{document}
\begin{titlepage}
    \begin{center}
    
    % school logo
    \includegraphics[width=0.9\textwidth]{ntu_logo}
    \\[6cm]
    
    % report title and author
    \uppercase{
    \textbf{CZ3005 Artificial Intelligence}\\
    \textbf{Assignment 2 Report}
    \\[2cm]
    \textbf{Nguyen Huy Anh}\\
    \textbf{U1420871B}\\
    \textbf{TSP5}
    }
    
    \vfill
    
    % Bottom of the page
    \textbf{School of Computer Science and Engineering}
    \\
    \textbf{Academic Year 2017/18, Semester 1}
    
    \end{center}
\end{titlepage}

\section*{Assignment 3: Subway sandwich interactor}

In this assignment, a Prolog-based interactor would be implemented to mimic
the Subway sandwich ordering experience. Customers would be able to customize
their order from a predefined range of options.

The implementation is split into two Prolog scripts, \texttt{subway.pl} which
stores the Subway model and \texttt{interface.pl} which implements the
interaction. The entry point to the implementation would be the predicate
\texttt{start/0} in \texttt{interface.pl}.

\subsection*{Model: \texttt{subway.pl}}

The model in \texttt{subway.pl} includes the lists of options available as well
as helper predicates \texttt{options/1} to display the lists and
\texttt{selected/2} to choose a list option.

\begin{lstlisting}
options_list([]). % empty list
options_list([H]) :- write(H), write('.'). % final list item
options_list([H|T]) :- write(H), write(', '), options_list(T), !. % normal list
options(breads):- breads(L), write('breads = '), options_list(L).
options(mains):- mains(L), write('mains = '), options_list(L).
options(cheeses):- cheeses(L), write('cheeses = '), options_list(L).
options(vegs):- vegs(L), write('vegs = '), options_list(L).
options(sauces):- sauces(L), write('sauces = '), options_list(L).
options(cookies):- cookies(L), write('cookies = '), options_list(L).
options(addons):- addons(L), write('addons = '), options_list(L).

selected(X, breads) :- breads(L), member(X, L), write('bread = '), write(X), !.
selected(X, mains) :- mains(L), member(X, L), write('main = '), write(X), !.
selected(X, cheeses) :- cheeses(L), member(X, L), write('cheese = '), write(X), !.
selected(X, vegs) :- vegs(L), member(X, L), write('veg = '), write(X), !.
selected(X, sauces) :- sauces(L), member(X, L), write('sauce = '), write(X), !.
selected(X, cookies) :- cookies(L), member(X, L), write('cookie = '), write(X), !.
selected(X, addons) :- addons(L), member(X, L), write('addon = '), write(X), !.

breads([italian, heartyitalian, wheat, honeyoat, wholegrain, parmesan, flatbread, wrap, salad]).
mains([ham, turkey, coldcut, bbqchicken, tuna, eggmayo, meatball]).
cheeses([american, monterrey, none]).
vegs([lettuce, tomato, cucumber, capsicum, onion, jalapeno, pickle, avocado]).
sauces([bbq, honeymustard, mustard, sweetonion, redwine, mayonnaise, chipottle, ranch, vinegar]).
cookies([chocchip, doublechoc, walnut, macnut, raspberry, brownie]).
addons([drink, soup, chips]).
\end{lstlisting}

As seen above, the implementation of \texttt{options/1} uses an underlying 
predicate \texttt{options\\\_list/1} to recursively \texttt{write} the list
elements; it handles the special case of 1-element list (ending the text by
using ``.'') and empty list (output nothing). The implementation of
\texttt{selected/2} unifies the list with a variable \texttt{L}, then perform
a member check and \texttt{write} the choice if true.

\subsection*{Interactor: \texttt{interface.pl}}

The interactor in \texttt{interface.pl} is written in pure Prolog, consisting
of a series of predicates and prompts. The root predicates are \texttt{help/0}
which prints out a small help text, and \texttt{start/0} which starts the
interactive session.

\begin{lstlisting}
start :-
    write('----------------------------------------------'),nl,
    write('--------------------SUBWAY--------------------'),nl,
    write('----------------------------------------------'),nl,
    exec,
    end.
\end{lstlisting}

The predicate \texttt{exec/0} is the analogous to a \texttt{main\,()} function for
this interactor. It asks the user to choose a meal option, either normal, veggie,
vegan or value; the corresponding \texttt{meal\_*/0} predicate is called and
\texttt{meal\,/1} is asserted. 

\begin{lstlisting}
exec:-
    write('Please choose your meal type (one of normal, veggie, vegan, value)?'),nl,
    read(Meal),
    ( (Meal == veggie) -> 
        write('meal = '), write(meal), nl,
        meal_veggie, assert(meal(veggie)) ;
        (Meal == vegan) ->
            write('meal = '), write(meal), nl,
            meal_vegan, assert(meal(vegan)) ;
            (Meal == value) ->
                write('meal = '), write(meal), nl,
                meal_value, assert(meal(value)) ;
                write('meal = normal'), nl,
                meal_normal, assert(meal(normal)) ), % normal by default
    write('----------------------------------------------'),nl,
    write('----------------------------------------------'),nl,
    display. 
\end{lstlisting}

Each \texttt{meal\_*/0} predicate is a collection of \texttt{query\_*/0} predicates
in a specific order; each corresponding to a list in the model.

\begin{lstlisting}
meal_normal :-
    query_bread, query_main, query_cheese, query_veg,
    query_sauce, query_cookie, query_addon.
meal_veggie :-
    query_bread, query_veg,
    query_sauce, query_cookie, query_addon.
meal_vegan :-
    query_bread, query_veg,
    query_sauce, query_addon.
meal_value :-
    query_bread, query_main, query_cheese, query_veg,
    query_sauce.
\end{lstlisting}

A \texttt{query\_*/0} predicate is implemented using a corresponding helper
predicate \texttt{query\_*\_loop\\/0} that handles error-checking, multiple
values and assertions.
Input errors occur when the user input is not in the corresponding list; the
predicate \texttt{selected/2} in the model would return \texttt{false} and a
re-input prompt is printed out. For certain lists (\texttt{vegs}, \texttt{sauces},
\texttt{cookies} and \texttt{addons}), there could be multiple values in an order;
a sentinel value of \texttt{0} when entered will end the input list. Sample
demonstrations are included below.

\begin{lstlisting}
query_bread_loop :-
    read(X),
    ( selected(X, breads) -> % if true then proceed
        nl, assert(bread(X)) ;
        write('Invalid option, try again!'),nl, % else try again 
        query_bread_loop ).
query_veg_loop :-
    read(X),
    ( not(X == 0) -> % if input is 0, proceed to next list
        ( selected(X, vegs) -> % if true then proceed
            nl, assert(veg(X)) ;
            write('Invalid option, try again!'),nl ), % else try again         
        query_veg_loop ;
        true ).
query_bread :- options(breads),nl,write('Please input your bread: '),nl,
    query_bread_loop.
query_veg :- options(vegs),nl,write('Please input your vegs (0 to end): '),nl,
    query_veg_loop.
\end{lstlisting}

The final predicate in \texttt{exec/0}, \texttt{display/0} would display the final
order made by the user. It is implemented using a helper predicate,
\texttt{choices/8} which collects the various choices from \texttt{query\_*/0}
predicates.

\begin{lstlisting}
choices(Meal, Bread, Main, Cheese, Vegs, Sauces, Cookies, Addons) :-
    meal(Meal),
    bread(Bread),
    main(Main),
    cheese(Cheese),
    findall(X, veg(X), Vegs),
    findall(X, sauce(X), Sauces),
    findall(X, cookie(X), Cookies),
    findall(X, addon(X), Addons).
display :-
    choices(Meal, Bread, Main, Cheese, Vegs, Sauces, Cookies, Addons),
    write('Your meal type: '),write(Meal), nl,
    write('Your choices:'),nl,
    write('Bread: '), write(Bread), nl,
    write('Main: '), write(Main), nl,
    write('Cheese: '), write(Cheese), nl,
    atomic_list_concat(Vegs, ',', Veg),
    write('Vegs: '), write(Veg), nl,
    atomic_list_concat(Sauces, ',', Sauce),
    write('Sauces: '), write(Sauce), nl,
    atomic_list_concat(Cookies, ',', Cookie),
    write('Cookies: '), write(Cookie), nl,
    atomic_list_concat(Addons, ',', Addon),
    write('Addons: '), write(Addon), nl.
\end{lstlisting}

Finally, the final predicate in \texttt{start/0}, \texttt{end/0} would end the run
and flush all choices from the system, using \texttt{retract/1}.

\begin{lstlisting}
flush :- retract(meal(_)), fail.
flush :- retract(bread(_)), fail.
flush :- retract(main(_)), fail.
flush :- retract(cheese(_)), fail.
flush :- retract(veg(_)), fail.
flush :- retract(sauce(_)), fail.
flush :- retract(cookie(_)), fail.
flush :- retract(addon(_)), fail.

end :-
    write('----------------------------------------------'),nl,
    write('------------------END-SUBWAY------------------'),nl,
    write('----------------------------------------------'),
    flush. % reset the system
\end{lstlisting}

\subsection*{Sample Interactor Run}

The following snippet presents a sample run for the interactor.

\begin{lstlisting}
$ swipl
Welcome to SWI-Prolog (threaded, 64 bits, version 7.6.2)
SWI-Prolog comes with ABSOLUTELY NO WARRANTY. This is free software.
Please run ?- license. for legal details.

For online help and background, visit http://www.swi-prolog.org
For built-in help, use ?- help(Topic). or ?- apropos(Word).

?- ['interface.pl'].
true.

?- help.
Enter start. to start the process.
true.

?- start.
----------------------------------------------
--------------------SUBWAY--------------------
----------------------------------------------
Please choose your meal type (one of normal, veggie, vegan, value)?
|: normal.
meal = normal
breads = italian, heartyitalian, wheat, honeyoat, wholegrain, parmesan, flatbread, wrap, salad.
Please input your bread:
|: italian.
bread = italian

mains = ham, turkey, coldcut, bbqchicken, tuna, eggmayo, meatball.
Please input your main:
|: hello.
Invalid option, try again!
|: ham.
main = ham

cheeses = american, monterrey, none.
Please input your cheese:
|: none.
cheese = none

vegs = lettuce, tomato, cucumber, capsicum, onion, jalapeno, pickle, avocado.
Please input your vegs (0 to end):
|: avocado.
veg = avocado
|: tomato.
veg = tomato
|: onion.
veg = onion
|: 0.

sauces = bbq, honeymustard, mustard, sweetonion, redwine, mayonnaise, chipottle, ranch, vinegar.
Please input your sauces (0 to end):
|: bbq.
sauce = bbq
|: 0.

cookies = chocchip, doublechoc, walnut, macnut, raspberry, brownie.
Please input your cookies (0 to end):
|: 0.

addons = drink, soup, chips.
Please input your addons (0 to end):
|: drink.
addon = drink
|: 0.

----------------------------------------------
----------------------------------------------
Your meal type: normal
Your choices:
Bread: italian
Main: ham
Cheese: none
Vegs: avocado,tomato,onion
Sauces: bbq
Cookies:
Addons: drink
----------------------------------------------
------------------END-SUBWAY------------------
----------------------------------------------
false.
\end{lstlisting}

\end{document}
